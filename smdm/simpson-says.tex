\documentclass[]{article}
\usepackage{lmodern}
\usepackage{amssymb,amsmath}
\usepackage{ifxetex,ifluatex}
\usepackage[numbers]{natbib}
\ifnum 0\ifxetex 1\fi\ifluatex 1\fi=0 % if pdftex
  \usepackage[T1]{fontenc}
  \usepackage[utf8]{inputenc}
\else % if luatex or xelatex
  \ifxetex
    \usepackage{mathspec}
  \else
    \usepackage{fontspec}
  \fi
  \defaultfontfeatures{Ligatures=TeX,Scale=MatchLowercase}
\fi
% use upquote if available, for straight quotes in verbatim environments
\IfFileExists{upquote.sty}{\usepackage{upquote}}{}
% use microtype if available
\IfFileExists{microtype.sty}{%
\usepackage{microtype}
\UseMicrotypeSet[protrusion]{basicmath} % disable protrusion for tt fonts
}{}
\usepackage[margin=1in]{geometry}
\usepackage{hyperref}
\hypersetup{unicode=true,
            pdftitle={Half Cycle and Life Tables: What does Simpson say?},
            pdfauthor={Shawn Garbett MS},
            pdfborder={0 0 0},
            breaklinks=true}
\urlstyle{same}  % don't use monospace font for urls
\usepackage{graphicx,grffile}
\makeatletter
\def\maxwidth{\ifdim\Gin@nat@width>\linewidth\linewidth\else\Gin@nat@width\fi}
\def\maxheight{\ifdim\Gin@nat@height>\textheight\textheight\else\Gin@nat@height\fi}
\makeatother
% Scale images if necessary, so that they will not overflow the page
% margins by default, and it is still possible to overwrite the defaults
% using explicit options in \includegraphics[width, height, ...]{}
\setkeys{Gin}{width=\maxwidth,height=\maxheight,keepaspectratio}
\IfFileExists{parskip.sty}{%
\usepackage{parskip}
}{% else
\setlength{\parindent}{0pt}
\setlength{\parskip}{6pt plus 2pt minus 1pt}
}
\setlength{\emergencystretch}{3em}  % prevent overfull lines
\providecommand{\tightlist}{%
  \setlength{\itemsep}{0pt}\setlength{\parskip}{0pt}}
\setcounter{secnumdepth}{0}
% Redefines (sub)paragraphs to behave more like sections
\ifx\paragraph\undefined\else
\let\oldparagraph\paragraph
\renewcommand{\paragraph}[1]{\oldparagraph{#1}\mbox{}}
\fi
\ifx\subparagraph\undefined\else
\let\oldsubparagraph\subparagraph
\renewcommand{\subparagraph}[1]{\oldsubparagraph{#1}\mbox{}}
\fi

%%% Use protect on footnotes to avoid problems with footnotes in titles
\let\rmarkdownfootnote\footnote%
\def\footnote{\protect\rmarkdownfootnote}

%%% Change title format to be more compact
\usepackage{titling}

% Create subtitle command for use in maketitle
\newcommand{\subtitle}[1]{
  \posttitle{
    \begin{center}\large#1\end{center}
    }
}

\setlength{\droptitle}{-2em}

  \title{Half Cycle and Life Tables: What does Simpson say?}
    \pretitle{\vspace{\droptitle}\centering\huge}
  \posttitle{\par}
    \author{Shawn Garbett MS\footnote{Employed by Vanderbilt University Medical Center,
  Biostatistics Department} \footnote{Financial support for this study was provided entirely by a grant from the 
  National Institute of Health (1R01HG009694-01). The funding agreement ensured the author's independence in designing the study, interpreting the data, writing and publishing the report }}
    \preauthor{\centering\large\emph}
  \postauthor{\par}
      \predate{\centering\large\emph}
  \postdate{\par}
    \date{September 6, 2018}

\usepackage[LGR,T1]{fontenc}
\usepackage[utf8]{inputenc}
\usepackage{textgreek}
\usepackage{float}
\usepackage[x11names,dvipsnames,table]{xcolor}
\usepackage{boldline}
\usepackage{colortbl}
\usepackage{hhline}

\begin{document}
\maketitle
\begin{abstract}
The half-cycle correction method and the recommended life table
membership methods are commonly used to numerical compute state
membership in discrete Markov models in the health economics field. The
essence of the problem is numerical integration over a fixed interval
size. This problem has a rich history dating back to Newton and Cotes
correspondences, and a better method was published in 1743 by Simpson
with accompanying proof about expected error of the method. The
alternative Simpson method is considered state of the art and easy to
use. We conclude that both the half-cycle correction and life table
method should be dropped in favor of the alternative Simpson method.
\end{abstract}

\section{Introduction}

Barendregt\cite{barendregt2009} recommended banishing the half-cycle correction in
summing state occupancy in a discrete Markov model, especially in the
field of health economics for cost and utility
calculations\cite{barendregt2009,naimark2008,sonnenberg1993}. The
recommended solution was to use life tables. The life table method is a
restated version of the \emph{trapezoidal rule} which is a closed
Newton-Cotes formula.

These methods are foundational in the field of numerical integration
(also called quadrature) which numerically computes the area under a
curve computed at fixed intervals. In the context of discrete Markov
models numerical integration is an estimator of state occupancy or some
derived function from that occupancy such as total quality-adjusted life
years (QALY) or total cost typically modified by a discounting function.
These methods all assume that the resulting function being integrated is
smooth and does not have discontinuities, which fortunately is common
for health economic models. These methods were widely used in the 18th
and 19th centuries for manual computation and a very common tool of
human `computers'\cite{epperson2007}. In this context it is worth
understanding the state of the art, its known error, and limitations.
The discovery of the trapezoidal rule has occurred independently in
other fields as well\cite{tai1994}.

A closed formula uses the end points of the data set and an open formula
only uses the interior points. The proposed life table method is none
other than the \emph{trapezoidal rule} as proposed by Sir Issac Newton:

\[ \int^{x_2}_{x_1} f(x) dx = \frac{h}{2}[ f(x_1) + f(x_2)] + O(h^3 f'') \]

This formula represents that integration can be approximated for a
function \(f\) over an interval, \(h=x_2 - x_1\), by evaluating the ends
points. The resulting error is proportional to the cube of the interval
size times the second derivative of the function inside the region. So
extending this over several intervals of an evaluated discrete Markov
model, the middle intervals weights add to 1 and the end points stay at
\(\frac{1}{2}\) weight. This is the life table method.

Such a theory is furthered by Simpson in 1743 to Simpson's rule, which
didn't have the clean cancellation like the trapezoidal rule, and
thus Simpson created his \(\frac{3}{8}\) rule with a clean set of
inner weights (all 1's) and numerical errors on the order of \(O(h^5 f^{(4)})\)\cite{simpson1743}.
Later Boole (Sometimes misattributed to Bode from an earlier
  misprint in Abramowitz\cite{abramowitz1965}) extended this further by
reducing the error another 2 orders\cite{boole1860}. At this point the
practice of naming these formulas after their discoverer stopped, but
tables of coefficients for higher order versions exist\cite{abramowitz1965}.
To get the coefficients to come out cleanly (all 1's in the middle), various alternations
of strategies have been explored, and one commonly used accepted for
general use in numerical integration for fixed intervals is known as the
\emph{alternative extended Simpson's rule}\cite{epperson2007, hamming1973, press1992}.

\[
\begin{split}
\int^{x_n}_{x_1} f(x) dx = & \frac{h}{48} [ 17 f(x_1) + 59 f(x_2) + 43 f(x_3) + 49 f(x_4) + f(x_5) + \\ 
  & \dots + f(x_{n-4}) + 49 f(x_{n-3}) + 43 f(x_{n-2}) + 59 f(x_{n-1}) + 17 f(x_n) ) ] + \\
  O\left(h^5 f^{(4)} \right)
\end{split}
\]

Using the resulting coefficients from this rule reduces overall
numerical error greatly over the trapezoidal rule. For an interval size
of one. the error is dependent on the 4th derivative of the function
instead of the 2nd. With well behaved functions, like those commonly
found in health economic models, the 4th derivative is much smaller than
the 2nd. Discounting is easily included by multiplying each time point
value by the appropriate discounting function.

It is worth noting the effect on error the interval size has. If
reducing the time step of a Markov model from yearly to monthly the numerical error reduces
more than 3 decimal places as
\((1/12)^3 \approx 0.0006\) for the trapezoidal method and 5 decimal
places for the alternate extended Simpson's method as
\((1/12)^5 \approx 0.000004\). Rescaling the transition probability
matrix of a Markov model is a strong method to reduce error and improve
accuracy under either method.

Many previous discussions have centered on explaining how the half-cycle
correction works. Simpson's approach doesn't have as simple an
explanation without understanding the underlying calculus, but it does
come with a mathematical proof of its properties. A basic conceptual
viewpoint of what is happening is that differences of the points can be
used to estimate the first derivative, and the differences of those
derivative estimates can be used to estimate the 2nd derivative, etc.
Each of these estimates provide more information about the shape and
curvature of the function being integrated and contribute differing
amounts to the final estimate. Hence, the weighting is determined by the
addition of all these contributions, and many terms combine or cancel
till a simple weighted sum is all that is required to provide an
estimate that achieves acceptable accuracy given the data.

One limitation of this method is that the error is amplified a bit when
encountering discontinuities in the integrated function. For example a
tunnel state occupancy has a sharp non smooth boundary when it
transitions from initial loading to a balance of inflow and outflow.
This can occur in Markov models where a tunnel state is initially
unoccupied. Further, these methods also assume that the function is not
``spiky'' such as is seen in spectrography data,
as this can further reduce numerical accuracy\cite{kalambet2018}.
Fortunately, most health economic models do not fall into this category
and the resulting functions are relatively smooth with the exception of
the single point at the loading of a tunnel state.


\subsection{Methods}

In R the alternate extended Simpson's method can be implemented as follows:

\begin{verbatim}
alt_simp_coef <- function(i) c(17, 59, 43, 49, rep(48, i-8), 49, 43, 59, 17) / 48
alt_simp      <- function(x,h) h*sum(alt_simp_coef(length(x)) * x)
\end{verbatim}

In Excel the following VB Macro works given a range of values:

\begin{verbatim}
Function Simpson(values As Array)
  
  Dim sum    As Double
  Dim begin  As Long
  Dim final  As Long
  
  begin = LBound(values)
  final = UBound(values)
  
  If final - begin < 8 Then
    Err.Raise 9,"Simpson()","Need 9 or more points to integrate"
  End If

  sum = 0.0
  sum = 17*values(begin  ,1)/48 + _
        59*values(begin+1,1)/48 + _
        43*values(begin+2,1)/48 + _
        49*values(begin+3,1)/48 + _
        49*values(final-3,1)/48 + _
        43*values(final-2,1)/48 + _
        59*values(final-1,1)/48 + _
        17*values(final  ,1)/48
   
  For i = begin+4 to final-4
    sum = sum + values(i,1)
  Next

  Simpson = sum

End Function
\end{verbatim}

\section{Results}

This is an extension of the Life Table example from Barendregt
with results compared to the alternative extended Simpson's rule\cite{barendregt2009}.
In this example, we will look at a population of 1000
individuals who are aging from 0 to 10, and we are interested in the
average total QALYs. The population under consideration has a constant
death rate of 0.52. The analytical truth of this model is easily
computed as follows.

\[
\int^{10}_0 1000 (1-0.52)^x dx = \frac{1000}{\log(0.48)}  0.48^x \Biggr|^{10}_0 \approx  1361.6
\]

This simple example in Table 1 shows the resulting numerical error is reduced by more
than an order of magnitude by switching from trapzoidal to Simpson's
method.

\section{Discussion}

In conclusion, the field of numerical intergration has long investigated
the problem of integrating a function over a fixed interval as provided
by the discrete Markov modeling method that is used frequently in health
economic models. The idea presented of using life tables
(or half-cycle correction) should be called by it's
mathematical name, the trapezoidal rule, and is acceptable to improve
the accuracy of estimates. However, if one is in pursuit of higher
accuracy, the trapezoidal rule should be passed over in favor of the
alternative extended Simpson's rule due to its superior numerical
accuracy. It's a simple weighted sum that is easily done with a
mathematical proof of behavior. In short, Simpson would recommend it.

\section{Tables and figures}



\definecolor{nejm-yellow}{RGB}{255,251,237}
\definecolor{nejm-header}{RGB}{247,244,239}

\begin{table}[H]
\centering
{\fontfamily{cmss}\selectfont
\rowcolors{2}{nejm-yellow}{white}
{\renewcommand{\arraystretch}{1.2}\begin{tabular}{|lcccccc|}
\hline
\rowcolor{nejm-header}\multicolumn{7}{|l|}{Barendregt Life Table Compared with Simpsons Method} \\
\hline
\textbf{Summary} & \textbf{Age} & \textbf{Living} & \textbf{Trapezoid} & \textbf{Lx} & \textbf{Simpson} & \textbf{Lsim}\\
\textbf{} & 0 & 1000 & 0.5 & 500.0 & 0.35 & 354.2\\
\textbf{} & 1 & 480 & 1 & 480.0 & 1.23 & 590.0\\
\textbf{} & 2 & 230.4 & 1 & 230.4 & 0.90 & 206.4\\
\textbf{} & 3 & 110.6 & 1 & 110.6 & 1.02 & 112.9\\
\textbf{} & 4 & 53.1 & 1 & 53.1 & 1.00 & 53.1\\
\textbf{} & 5 & 25.5 & 1 & 25.5 & 1.00 & 25.5\\
\textbf{} & 6 & 12.2 & 1 & 12.2 & 1.00 & 12.2\\
\textbf{} & 7 & 5.9 & 1 & 5.9 & 1.02 & 6.0\\
\textbf{} & 8 & 2.8 & 1 & 2.8 & 0.90 & 2.5\\
\textbf{} & 9 & 1.4 & 1 & 1.4 & 1.23 & 1.7\\
\textbf{} & 10 & 0.6 & 0.5 & 0.3 & 0.35 & 0.2\\
\textbf{Sum} & 55 & 1922.5 & 10 & 1422.2 & 10.00 & 1364.7\\
\textbf{Absolute Error} &  &  &  & 60.6 &  & 3.1\\
\textbf{Percent Error} &  &  &  & 4.5 &  & 0.2\\
\hline
\end{tabular}}
}
\caption{Life Table Compared with Simpson's Method}

The Age column is the age of the participants in the population of 1000
over the model run. Living represents the number of living individuals
in the model at that age. The Trapezoid column is the weights from the
trapezoid rule (or life table method). Lx represents the resulting
contribution to the summation. The Simpson column is the weighting of
the points in the alternative extended Simpson's rule followed by the
Lsim which is the total contribution of living to the total life in the
model. There are rows showing the summation of the column, and the
resulting error from the known analytical truth of the model.

\end{table}

\section{Acknowledgments}

The author would like to thank John Graves and Lisa McCawley for editorial support in preparing this manuscript.

\section{Declaration of Conflicting Interests}

The author declares that there is no conflict of interest.

\section{References}

\begin{thebibliography}{99}

\bibitem{abramowitz1965}
Abramowitz, Milton, and Irene A. Stegun. Handbook of
Mathematical Functions: With Formulas, Graphs, and
Mathematical Tables. 1st edition. Dover Publications; 1965.

\bibitem{barendregt2009}
Barendregt, Jan J. ``The Half-Cycle Correction: Banish Rather Than
Explain It.'' Medical Decision Making. 2009 Jul 1;29(4):500--502.
doi:10.1177/0272989x09340585.

\bibitem{boole1860}
Boole, George. A Treatise on the Calculus of Finite
Differences. Macmillan Co. London; 860. 

\bibitem{epperson2007}
Epperson, James F. 2007. An Introduction to Numerical Methods and
Analysis. 1st edition. Wiley-Interscience; 2007.

\bibitem{hamming1973}
Hamming, Richard W. Numerical Methods for Scientists and
Engineers. Dover Publications; 1973.

\bibitem{kalambet2018}
Kalambet, Yuri, Yuri Kozmin, and Andrey Samokhin. ``Comparison of
Integration Rules in the Case of Very Narrow Chromatographic Peaks.''
Chemometrics and Intelligent Laboratory Systems. 2018 Aug; 179:22--30.
doi:10.1016/j.chemolab.2018.06.001.

\bibitem{naimark2008}
Naimark, David M. J., Michelle Bott, and Murray Krahn.``The
Half-Cycle Correction Explained: Two Alternative Pedagogical
Approaches.'' Medical Decision Making. 2008 Sep;28(5):706--12.
doi:10.1177/0272989x08315241.

\bibitem{press1992}
Press, William H., Brian P. Flannery, Saul A. Teukolsky, and William T.
Vetterling. Numerical Recipes in c: The Art of Scientific
Computing, Second Edition. Cambridge University Press; 1992.

\bibitem{simpson1743}
Simpson, Thomas. Mathematical Dissertations on a Variety of
Physical and Analytical Subjects. T. Woodward; 1743.

\bibitem{sonnenberg1993}
Sonnenberg, Frank A., and J. Robert Beck. 1993. ``Markov Models in
Medical Decision Making: a practical guide.'' Medical Decision Making. 1993 Oct; 13(4):322--38.
doi:10.1177/0272989x9301300409.

\bibitem{tai1994}
Tai, M. M. ``A Mathematical Model for the Determination of Total
Area Under Glucose Tolerance and Other Metabolic Curves.''
Diabetes Care. 1994 Feb;17(2):152--54. Diabetes Care.
doi:10.2337/diacare.17.2.152.

\end{thebibliography}

\end{document}
